
%\documentclass[man]{apa}
%\documentclass[doc]{article}{styles/apacls/apa}
\documentclass{report}


\usepackage{mathptmx}       % selects Times Roman as basic font
\usepackage{helvet}         % selects Helvetica as sans-serif font
\usepackage{courier}        % selects Courier as typewriter font
\usepackage{type1cm}        % activate if the above 3 fonts are
                            % not available on your system
\usepackage[utf8]{inputenc}
\usepackage{textcomp}
\usepackage{makeidx}         % allows index generation
\usepackage{graphicx}        % standard LaTeX graphics tool
                             % when including figure files
\usepackage{multicol}        % used for the two-column index
\usepackage[bottom]{footmisc}% places footnotes at page bottom

\usepackage{verbatim}        % per comentaris multilinia
\usepackage[utf8]{inputenc} %permite escribir {\'a},{\'e},{\'u},{\'o},{\'\i} & {\~n} directamente
\usepackage[british]{babel}
\usepackage{csquotes}
\usepackage{amsmath, amsthm, amssymb}


\begin{document}
\title{Coevolutionary strategies in MultiAgent systems. An approach using socionatural realistic environments.\\
\small{Master Thesis}}
\author{Alexis Torrano Mart\'inez}
\maketitle
\newpage

\pagenumbering{roman}
\begin{abstract}
The aim of this master thesis is the development of a multiagent model for a simulation  of two populations whose interactions are strongly influenced by a realistic landscape.
This research will be in line with Consolider-Simulpast (www.simulpast.es), an interdisciplinary project aimed to create simulations designed to be used in archaeological studies of human-environment interaction, decision-making processes and coevolutionary/competition behaviours of past societies.
The work plan will be focused on the development of first-stage models for two societies in the age of agriculture spreading surpasing the hunting and foraging way of living. The simulation will involve a climate engine for seasonality depending primarily on variable rainfall rate. Landscape information will be created from satellite image rasters. Constants, and variable relationship shall be modelled from measures and interviews with the experts. Data analysis tasks will be undertaken to validate the models and detect patterns in the archaeological record. Furthermore a comparison will be stablished between the classical simple models used in social simulation[1][2][8] and more advanced approaches.
\end{abstract}
\newpage 

\setcounter{tocdepth}{6}
\tableofcontents
\newpage 

\pagenumbering{arabic}
\chapter{Introduction}

\section{Description}

%Problem, Gujarat, Archeology (com al resum de la matri­cula).

\section{Motivation}

Simulation has been following an evolution in the models and paradigms applied to represent its target
systems. Dynamical systems, differential equations have been used and an overall simplification of the parts of the systems in the history of simulation to operate with the abstraction and simplification of the problems.
Mainly, reusing ideas from physical simulations, social sciences has modelled complex systems with dynamical
atomic entities that apply a simple set of centralized rules to move around the environment modelled. It was seen that due to the deeper details in human behaviour the question that could be solved and asked to that kind of models could not go very further as expected. 

To solve non-linearity phenomena, heterogeneity, hysteressis and other issues typical of complex systems, Agent Based Models(ABMs) where introduced to gain more insight of the modelled systems achieving good results. But under 
some conditions of very complex relationships between agents, highly specialized decission taking procedures
and issues in the environment led agents to need a sophysticated reasoning and problem solving capabilities
that are not specified and not yet introduced in ABMs coming from Social Sciences. 

We have found an example of that situation in our case study located in Gujarat from Simulpast project. Our agents must interact with a regular but changing environment to get resources, plan its actions, coordinate with its group and compete with other groups in a \textbf{co-evolution} dynamic. Because we want to find out why the Gujarat HunterGatherer(HG) way of life lasted more than in other place of Earth in its competition against AgroPastoralist(AP), we need to embed the behaviours of the survival strategies used by these groups. A simply reactive agent cannot cope with short term plus long term decissions in that competitive environment. So the question stated as topic for this Master Thesis project is \textbf{``do we get better results in social sciences simulations adding deeper AI techniques to make richer the behaviour or decission making engines of the entities in the system?''}. As we understand ``better results``, the outcome from the use of AI should be a more sound validation of the model, a nearier match of modelled behaviours with the real ones, and clearer, richer and robust scientific conclusions.

In order to study such posibilities Sugarscape[SugarScape] is a good framework to extend. Sugarscape is an artificial society developed by Joshua Epstein et al[Epstein]. where a number of inhabitants move to collect resources they need to live. Sugarscape models perception, lattice scanning in search of resources, sexual reproduction of the agents in the simulation, market relationships, immunology and spreading of diseases, and feature evolution. Epstein analises different experiments executed in the Sugarscape offering his conclusions and the dynamics emerging from the simulations. The results and conclusions will feed our AI experiments in order to make the comparisons of classical SugarScape agents vs AI agents, therefore, giving an answer to the topic of the Master Project.

	\section{Simulation}

%brief description
%hi ha altres aplicacions de simulacio on no hi ha component temporal i no ho estic esmentant! MonteCarlo,etc...
Simulation is a discipline for performing virtual experiments in a computer. Computational techniques are used to build a model that represents your system. The dynamics of that system is codified in an algorithm that computes a calculus imitating the changes of state in the model, hence having a representation of that change along the time of the system modelled. Simulate is to play to ''what happens if...?'', and it is aimed to discover and explain the dynamics of a system to enhace or guide strategy development, decission taking, management, solving problems without analytical solutions or knowledge discovering and research. Although, we could get other positive benefits from it like theory checking or training through the inmersion in virtual worlds responding to our input.

% purpose, a priori, outcome,  { deduction, induction, abduction(Alexis) } Axelrod's
% crec que els agradara, marketing, saborcillo de IA impregnant els racons
A simulation obeys some direction of experimentation, so a question must be set to drive the selection of features to model from the real system and give a direction to the modelling and the experiments design.These assumptions choice will prune the details not related to the questions to solve. It is not just for the sake of simplicity but for the practical reason that a model too near of the real system will be as hard as the original system to analyse.\\
Simulation, like deduction, starts with a set of those explicit assumptions. But unlike deduction, it does not prove theorems. Instead, a simulation generates data that can be analysed inductively. Unlike typical induction, however, the simulated data comes from a rigorously specified artificial experiment rather than direct measurement of the real world. While induction can be used to find patterns in data, and deduction can be used to find consequences of assumptions, simulation modeling can be used as an aid intuition and hypothesis validation tool ( also as space search mechanism for parameter tunning or optimization ).
%abduction
Just like in a tipical Sherlock Holmes adventure you pick the evidences, scenario for the experiment, and the common knowledge (initial expert assumptions about the model). You enter in a refinement cycle where you test hypothesis and readjust them to discover the theory, the ``plot``, the explanation of what is happening. Following the abductive reasoning schema ... matching with SSSS needs...
%--: simulation, results, validation and statement of your inital hypothesis and question...
%--: simulation connected to abduction
% connect abduction with the simulation lifecycle 

Social simulation is a research field that applies computational methods to study issues in the social sciences. The issues explored include problems in psychology, sociology, political science, economics, anthropology, geography, archaeology and linguistics \cite{TakahashiSallachRouchier2007}.\\
Social simulation aims to cross the gap between the descriptive approach used in the social sciences and the formal approach used in the hard sciences, by moving the focus on the processes/mechanisms/behaviors that build the social reality.\\
In social simulation, computers supports human reasoning activities by executing these mechanisms. This field explores the simulation of societies as complex non-linear systems, which are difficult to study with classical mathematical equation-based models.
%reducibility 
Most of the times, studying complex systems implies to cope with non reducibility. One of the examples is Gravitational Dynamics. If our assumption is the use of Newton's mechanics, we can predict the state at any time or not, depending on the scenario. For a one dimension world you can predict the state at time $t_n$ from the initial state $t_0$ without computing all the preceding ones. For two and more dimensions you can only compute directly state $t_n$ if less the three bodies are implied. So in a real environment of many bodies in a 3D world you need to compute all the states from the initial to the one you consider as the last one. The system is non analytically reducible and you are forced to apply simulation to visit all the states and develop the behaviour of the model.
% ( Classical Physics formulae used are non linear, acceleration is a quadratic factor in position <-- revisa aquesta fantasmada ) 

%emergence
Other of the main issues in complex systems simulation is emergence. While the initial assumptions may be simple, the consequences may not be at all obvious. The large-scale effects of locally interacting entities are called "emergent properties" of the system. Emergent properties are often surprising because it can be hard to anticipate the full consequences of even simple forms of interaction. 

%why simulation? because I need this non deducible emergence and the non deducible end state
There are some models, however, in which emergent properties can be formally deduced. Good examples include the neo-classical economic models in which rational agents operating under powerful assumptions about the availability of information and the capability to optimize can achieve an efficient reallocation of resources among themselves through costless trading. But when the agents use \textbf{adaptive} rather than optimizing strategies, deducing the consequences is often impossible; simulation becomes necessary.
% i ara estic preparat per conectar amb Social Science Simulation : reducibility + emergence forces to use simulation in S.S.


% Applying Simulation in the framework of Social Sciences to approach to the hypothesis of Simulpast targets, in concrete, about the case study from Gujarat, CS1.
\\
%simulacio de sistemes complexes : f\'isica de particules, mercats, ... ecologia i... social sciences...
\\
%n-body problem
%  2 body : formula
%  n body in 2D  : formula
%  n>2 body in 3D : simulation, non reducible
%   gravitation, electrostatics : input=initial position, output=end position, classical physics formulae: non linear!!!
%   acceleration is a quadratic factor in position!!!
\\
%mes endevant cal caracteritzar la simulació : deterministic/stochastic, static/dynamic, open/closed, %linear/nonlinear, hysteressis, stable/unstable,
%stationary stage, emulation/monteCarlo/traceDriven/DiscreteEvent/DiscreteUnitTimeStep
\\
\\
\\
\\
\\
	\section{Question} 

% trobo a faltar m\'es xixa te\`orica, aix\`o s'hauria de semblar m\'es a un cap\'itol del Norvig.
In classical simulation approaches, specifically in the branch of Social Simulation, active entities which model human actors are designed with very simple behaviour engines. The classical hypothesis is that a complex mind for entities in the simulation are not that needed and maybe even could lead to difficult analysis of final results of the simulations (too daring statement?). 

Our statement is that, on the contrary, the mind engine of a simulation entity should not be bounded to that limit but special attention must be paid to give any necessary sofistication to give the entity a correct behaviour, real enough, sensible to the changes in the environment and competent to solve the issues that will have to solve along its lifetime in the system. Even more, we think that this entities' capability to respond with complex behaviours is the core that roots the modelling granularity needed to catch the essential of the social systems that we want to model.( a l'apartat de ABMs ho tornem a dir però afegint que cal adaptabilitat, resposta no lineal, aprenentatge depenent del temps, histeresis,...).\\
( i aixo motivaria els ABMs) Applying such premises we will explore the possibility to give or enhance decission
making, problem solving capabilities to the entities with the aim to get more accurate simulations and realistic models with higher matching against our job hypothesis and premises. We will take the framework of ABMs to integrate the AI techniques in a decission making schema of action-response dynamics sensible to a modelled world(por los pelos...).\\

ABM -> Decentralization of Decision-Making
\\
exposicio mes detallada dels punts febles que creus que trobaràs als simple agents (i a les conclusions d'AI-Sugarscape se't confirma 
i d'altres outcomes)
\\
			\paragraph{Do AI techniques contribute to better simulation results?}
			\paragraph{Classic Simple Agent approach vs Rich Agents}
			\paragraph{Did Gujarat extreme enviromental conditions delayed the HG disappearence?}
\\
\newpage 
\chapter{Methodology}


% on poso el lifecycle del modelling procedure????
% specificacion --> expert's knowledge --> modelling --> prototipe --> { accept model | go to a previous state }
% revisiting early stages to refine models : toy model --> reviewing your assumptions about the main purpose
% interviews and ecotono meetings to motivate and help knowledge extraction
% ecotono modellings as a means to gain insight of the problem and the system, inspirational ideas.


\section{Intro}

metode = modelling\\
tecnica en ciencies socials = ABMs ( afegir capitol llibre Rub+Cris)
\\
que es modelling en ciencies socials 
\\

%%% XR

Computer modelling and complex systems simulation have dominated the scientific debate over the last decade, providing important outcomes in biology, geology and life sciences, and resulting in the birth of entirely new disciplines (e.g. bioinformatics, geoinformatics, health informatics, etc.). In the social sciences, the number of groups currently developing research programs in this direction is increasing. The results are extremely promising since simulation technologies have the potential to become an essential tool in the field \cite{Gilbert2008}.\\
However, some social scientists are sceptical about the idea of reproducing “inside” a computer population dynamics, because of the perceived complexity of social structures. This scepticism is understandable given the low number of projects that used this approach and the lack of experience of social scientists with these tools. Nevertheless, the research done in complexity science during recent years shows the way computer simulation can be applied to this field. Artificial intelligence portrays how the appropriate interconnection of very simple computational mechanisms is able to show extraordinary complex patterns, and access to distributed computing has become affordable. For this reason, agent-based simulation allows the implementation of experiments and studies that would not be viable otherwise \cite{Pavon2008}. \\

\section{Why model}

Modelling will be our framework for communication between archeologists' and sociologists' knowledge and their conceptualizations with our formal representations from computer science practices ( simulation, algorithms, AI ).\\
%què és modelar?
The concept of modelling is widely extended. It comes from the natural observation of the world and the curiosity or need to reproduce it.
As Epstein \cite{Epstein2008} says \textit{“Anyone who ventures a projection, or imagines how a social 
dynamic—an epidemic, war, or migration—would unfold is running some model”}. The challenge is to write it down, to turn it from \textit{implicit} where
assumptions or data are hidden to \textit{explicit}. It does not matter if the \textbf{mother} implies a mathematical formulation or any kind of graphical representation. Models are approximations to reality for an intented used \cite{Pidd2010}. Pidd proposes a graphical representation of a model as a box with inputs and outputs. The box will be black or grey depending on the purpose of the model. If one wished to perform controlled experimentation as in some areas of physical science, the box will be black since the model will be analysed through its outputs under defined inputs. On the contrary, if the box is grey that means we have some knowledge of the model's interior processes. That is the most suitable case when investigating case scenarios answering ``what if'' questions. The analyst part should be studied and it could display some unexpected emergent behaviour, a consequence of the internal dynamic interactions between the variables in the system. \\

%el model com a eina de treball en grups multidisciplinars
In social sciences and humanities, models are often expressed through natural language, which is ambigious. This is due to the nature of systems involved in their studies, which are often very complex in reasoning and suppositions. Moreover, each research discipline has its own vocabulary and approach. Therefore, scientists need to express the phenoma and ideas under study in the same rigorous manner \cite{Leeuw2004}. That is, they need to \textit{formalize} their models. Here when we talk about formal models we refer to the mechanism which permit us to understand, specify and analyse a system. Easily one can see the advantage on doing so when thinking in teams formed by experts from different areas of science. Models provide the way to make these multidisciplinary (pluridisciplinary? interdisciplinary?) teams work.\\

%%%% borrar? 
% com encaixa un model dins un estudi de simulació?
In this way, it is not ventured to say that modelling is one of the keys to do research. A model has impact in all aspects of a simulation study. 
In simulation field, the process of abstracting a model from a real or proposed system is called \textit{conceptual modelling} \cite{Robinson2008}. 
The resulting structure of concepts and views under which the simulationists guided for the development of a simulation model is called 
\textit{conceptual framework} \cite{Balci1988}. Serving as a guide, it facilitates the implementation and significantly reduces its complexity.
However, the conceptual modelling process is independent from the later implementation of the simulation.
In Figure ~\ref{fig:CM}, Heath \cite{Heath2009} shows how conceptual modelling is embebbed in a simplified simulation development process.
As we can see, the previous step of formalizing a model is to formulate a problem and the objectives of the simulation study.\\

%%% OK
%com cal modelar
To make the abstraction process effective an appropiate simplification of reality is needed \cite{Pidd2003}. That is, we need to set the 
boundaries of the real world portion we want to model in an appropiate that can give answer to the question we want to make. 
However, a model should be complex enough to answer the question raised \cite{Banks1998}. For example, a model that emulates a vehicle routing
problem can answer question on how a company should distribute its products in a given network. However, this model can not answer questions
on how this distribution will impact on the current traffic of the network. If we were interested on getting information on the traffic impact,
we should enlarge our model to model the traffic flow to calculate how a given distribution of vehicles will affect it.  \\

%%% OK
%el balanç entre simple i complex
Thus, we need to find a balance between real world and the conceptualized system. We could directly consider the most complex model to do an study
but we will encounter several problems, being the most important its credibility. How could we be certain that the non necessary components in 
our model are not affecting the results? As Robinson \cite{Robinson2008} states simple models have many advantages, such as they are faster, require 
less data, are more flexible and, more importantly, if we better understand them we can better interpret their results.
In fact, a good modelling design enhances the probability of simulation study success. However, not in all research areas simplicity is seen
as a positive value. For instance, Leeuw \cite{Leeuw2004} says archaeologists can not presume a simple behaviour until there is some evidence 
of it. As Davies \cite{Davies2003} states one should be carefull to simplify certain natural processes 
since it presumes certain assumptions about how they work so one could miss some important facet to explain it. The tendency however is to build
KISS (Keep it simple, stupid) models which is stems from Occam’s razor: the idea that things should be kept as simple
as possible and made as little more complex as explanation purposes demand \cite{Axelrod1997a}. Applied
to social simulation, KISS ideally seeks simple and abstract models that are general enough to be explanatory for multiple specific cases.\\

%models que es contraduien-- camp de la física p.e.

%tradició de modelat en ciències socials


\section{Modeling in social sciences}
\label{sec:modelsinSC}


certs apartats de l'ODD : adaptar les descripcions dels apartats de l'ODD

%% 3. Què cal modelar en sistemes socials?
%%	3.1 Def d'un agent
%%		Unitat de modelat
%%		Característiques
%%		Comportament i presa de decisions
%% 	3.2 Interaccions socials / relacions i estructura social
%%	3.3 Comportament espaial i entorn
%%	3.4 Comportament emergent

% \item A background on conceptual modelling in social sciences will be examined, stressing on the need of modelling human processes. The 
%difficulties social scientist often encounter when defining their models will be also pointed out. This will require a review of literature
%on the modelling process in social science. Moreover, a summary of different approaches will be 
%given for areas such as economics, political science, history, anthropology, demographics, anthropology, biology and ecology.


%Comp emergent

%Dificultats en la modelització en ciències socials: models matemàtics, models estadístics



ABM van be per a fer toy model, quan vols anar mes enlla i ser mes real, fa aigues --> cal IA.

discussio AI vs fake-AI .... si m'enrecordo dl q hem parlat.

L'intere\'s d'aplicar AI es evitar hand-made minds. Hand-made mind pot portar un bias introduit pel dissenyador.
Les opcions estan limitades als casos contemplats (disseny amb regles, xarxa semà ntica,.... posa mes exemples).
Es busca una aproximacio mes SOFT i no tant hardwired com un sistema de regles (argumenta-ho MOLT,
estaran els profes de la FIB, ¿Miquel Sanchez? has de rebatre molts paradigmes : Neurons, SVMs, CBRs, Production Rules,...). 
Menys bias, mes adaptatiu/flexible,... emergent i planificat.(llegeix llibres d'AI+SoftComp. ¿¿Copiar justificacions de la SoftComputing + justificacions de ProblemSolving???)

Logic Programming, Semantic Networks : hard to represent the foraging expertise of H.G. -> reduce the problem to 
resource adquisition -> math modelling -> maximization -> planning.

models -> dialog between data and ideas?

all models are wrong, but some of them are useful

models are formal ( we use formal languages to specify them, is that the reason? )
	atoms
	atom relationship
	syntax \& semantics
		trying to construct unambiguous 
	detect flaws and intuition mistakes
		posar l'exemple del graph (1r pdf modelthinking, 4 nodes vs 5 nodes, 3*1 > 4 * 0.71)
	
	common language with the community
			share ideas
			criticizing
			advices, redesign loop

	model is a logical/conceptual prototipe
		build a plane, first build a DaVinci/wood plane, build a tiny plane with a dummy
		to see ``what happens if I do things this way?"
	
Jean Lave \& James March, why model? : explain+predict phenomena, predict new phenomena, build+design systems.
(introdueix els tecnicismes PROGNOSIS, DIAGNOSIS)

R1Epstein.pdf(modelthinking){
Sixteen Reasons Other Than Prediction to Build Models

1. Explain (very distinct from predict)
2. Guide data collection
3. Illuminate core dynamics
4. Suggest dynamical analogies
5. Discover new questions
6. Promote a scientific habit of mind
7. Bound (bracket) outcomes to plausible ranges
8. Illuminate core uncertainties.
9. Offer crisis options in near-real time
10. Demonstrate tradeoffs / suggest efficiencies
11. Challenge the robustness of prevailing theory through perturbations
12. Expose prevailing wisdom as incompatible with available data
13. Train practitioners
14. Discipline the policy dialogue
15. Educate the general public
16. Reveal the apparently simple (complex) to be complex (simple)
}


Neurons, SVM, CBR : d'on trec els exemples per a fer learning? no tinc etiquetes concretes, categories, classes expli­cites ->
->Reinforcement Learning, aplicable?
com introduir MDP? (potser mes endevant et pots extendre sobre el que es MDP)

Uncertainty \& Imprecision : state of resources(other foragers, climate stochacity),uncomplete know,... 
modelthinking{The complexity wrought by the increases in information, adaptability, and interconnectedness implies a lack of predictability about what's next}.

modelthinking{modelling helps create a dialogue between ideas and data}.

Logic Programming, Semantic Networks : hard to represent the foraging expertise of H.G. -> reduce the problem to 
resource adquisition -> math modelling -> maximization -> planning.

no oblidis dir que si enriquim un model, l'estem fent més complexe. Afegir tots els detalls
converteix el model en una cosa tant difícil d'entendre com l'objecte a modelar.

modelling aporta un rigor al treballar amb les ideas del sistema observat que fa mes facil detectar errors, ambiguetats de les descripcions o especificacions. 

\section{Conceptual Framework}{Social Systems Modelization}

conceptes\\
---sistemes\\
------sistemes complexes\\
---------coevol \\
---------emergence\\
---------agents, una component que dispara la no linearitat i per tant la complexitat\\
---------heterogeneitat : dos \`atoms de Carboni es comporten igual en mateixes condicions, les persones no.\\

	\section{ABM}

ABM's. ABM's should be seen more as a metaphor (es una tecnica, no una metafora) than a methodology.

An ABM is build from some atomic identified entities in a bottom up design procedure.\\
Such entities are active decission making actors in the modelled system. The modelling lifecycle of an ABM will consider a stage where decission making processes must be identified from the system. Usually that decission making 
actions are carried along by more or less clear individual entities from the system. The modelling process will take the task to set the matching between these entities and the agent that will conform the ABM. The concept 
'' agent '' condenses a set of features that will specify the modelling metaphor that an agent represents:
some enclosed set of mechanisms to be aware of the state of the system, a set of goals to accomplish and the engine to decide from a bounded perception of the world, the actions to apply on it to achieve these goals. Besides the reasoning component of the agents, ABMs have a strong component of inter-agent relationship. How one agent interacts with other agents could as important or more as how it reacts to world changes.
\\
bottom-up approach, reductionist way of looking at the things, we try to understand the parts\\
to join them from their relationship and then we try to get the big picture.
\\
		%explica els agents com son a un ABM ( hauria de ser diferent d'un agent IA-Norvig que veiem mes endevant )
		% busca models on puguis veure com son els seus agents, Journals, Hidden Order (llibre, el te la Cris)  
\\
		Features:
\\
		  agents ( els features isolats )
		    goal planning/seeking
		    reasoning
		    adaptation
		    decission making
\\
		  social interaction
\\
		  emergence
		    complex behaviour patterns arising from goals and social interactions.
		    counterintuitive patterns : example of the colum in front of the exit door.
\\		    
		  bottom-up modelization
		    identify reasoning/deciding units : what will become agents
		    identify agents' relationships. 
\\
		  natural correspondence/matching of identified entities in social systems vs 
		  agents in the ABMs theoretical framework
\\
		  temporal correlation - histeresis
\\
		ABM vs other solutions
\\      
		    other solutions miss some points that ABMs can cover.
\\
		      non linearity, discontinuity, discrete spaces
		      non-homogeneous populations (different behaviours, HG,AP, mixed-HG-AP...)
		      unstability, critical points, transition points...
\\
		      DiffEq, as an example, fluid dynamics is an abstraction of a particle system.
		      You should justify that the abstraction from particle to fluid is sound and
		      parallel to the person->fluid abstraction, persons are not particles ;P
\\
		      com la fisica ha tingut la seva matematica molt avançada, les CienSocials
		      reusen molt de les metafores i analogies que fa la física de la matematica
		      als seus sistemes -> buscar models físics que no son sound ( busca publicacions,
		      analitza model de Fort, critica'l) -> processos amb cultura/expansio d'idees o espai de temps
		      diferents no es poden modelitzar amb física. 
\\
		    saber ilustrar quan usaries un ABM i quan no.
\\
\\
		    %aquests punts servirien per a justificar també perque triem ABMs i no un
		    %altre paradigma per a resoldre la nostra simulació
\\
		ABM issues
  
		    result analisys
		    computational costs


		\subsection{Evolution}
		Darwinisme : survival of the fittest.
		Ingredient afegit per a testejar l'adaptabilitat i el fitness de les dues estrategies : HG vs AP
		Dins el mark evolutiu establirem la competencia entre les dues estrategies : qui pugui tenir una
		millor supervivencia del seu offspring perpetuara  la seva estrategia (ADN)
		
		\subsection{Coevolution}
		Dos grups responent al medi. Pressio Evolutiva. El que fa un afecta a l'altre, aixo provoca una
		resposta d'adaptabilitat. Estudiem el feedback evolutiu entre HG,AP, i medi.

			%\subsubsection{Evolutionary \/ Genetic Programming?}
	
	\subsection{ODD.Agents} % i la resta? ODD.world = {ODD.climate, ODD.terrain, ODD.resources,....}


gradacio 
|..................................|
FISICA -> ABM / Teoria Jocs -> IA-MDP-Agents <<-- i aixo motiva el seguent apartat : IA Agents

Teoria de jocs costa de lligar amb dades reals i denses.


posar el tipus de t\`ecniques per donar IA als agents : mira els llibres de AI for games : rules, bayes,...

	\section{Intelligent agents} % explico un agent desde la IA ( Norvig )

	  .- reactive, proactive...
	  .- adaptation
	  .- reasoning
	  .- planner

	  \subsection{Planners, ...}

	    UCT/MDP
	    policies
	    actions
	    sectors

	    approaches : greedy, plan next action, ?`adding lookahead?



	   \chapter{Platforms / Software packages}
		\subsubsection{NetLogo...}

		    rasters -> +- automata
		    paralelism = 0
		    math libraries = 0?
		    ML, AI,... ???
		    -- efficiency
		    why do you not use netlogo

		\subsubsection{Pandora/Cassandra}

		Pandora : C++, STL, Python API
		Parallelism %transparent a l'usuari a traves de runtime generated code
	      
		The software we will use to implement the model is the Pandora Library, created by the social simulation
		research group of the Barcelona Supercomputing Centre. This tool is designed to implement agent-based 
		models and to execute them in high-performance computing environments (Rubio and Cela, 2010). It has been 
		explicitly programmed to allow the execution of large-scale agent-based simulations, and it is capable of 
		dealing with thousands of agents developing complex actions. The tool used has full GIS support to cope 
		with simulations in which spatial coordinates are relevant, as in the case here, where we want to detect 
		and compare spatial patterns. This library also allows the researcher to execute several simulations by 
		modifying initial parameters, as well as to distribute particular executions with high computer costs by 
		using a computer cluster. A cluster is formed by different linked computers (called nodes); the distribution 
		divides the computing cost of the execution between different nodes, each of which executes a part of the 
		entire simulation. As a result we will be able to run the simulation in a fraction of the time that would 
		be needed if we were using a single computer. The results of each simulation are stored in hierarchical data 
		format (HDF5), a popular format that can be loaded by most GIS. This feature is particularly useful, as we 
		will also use GIS to analyse simulation results.

		Finally, Pandora is complemented by Cassandra, a program developed to analyse the results generated by a simulation created
		with the library. 

		why do you use Pandora

\newpage 
\chapter{SugarScape vs Advanced Sugarscape} %% aqui veig mes clar que tota aquesta part funcionaria més com un ODD

\\
\\
 .. as proof case study
%SugarScape(Epstein) : environment for resource retrieval, social interactions, reproduction
%Match results, AI-agents are more proficient than their nonAI-counterparts
\\

%		Perque comportaments + rics? l'approach classic es queda curt quan compliques les coses
		(add all the good features) (justifica-ho!, en que i com es queda mes curt?)
      
%		A question:
		  
%		  1.- do we get best results in our simulations adding deeper AI techniques to make richer
%		  the behaviour or decission making engines of the entities in the system? How should this be
%		  done?

%			\textbf{Sugarscape as test example for this hypothesis/topic}
		
%		  això ve del CS1... i ara us l'explico ( la pregunta 2)


%		  ( la 2 es mes per posar en contexte, la 1 es el topic, nomes existeix 1 pregunta)

%		We want to apply the trend begun with the announced motivation to the case study of Gujarat in
%Simulpast project. 	
		  Why HG survived more than expected? ( l'aplicació de la pregunta 1)
			\textbf{interaction society vs envirm}
				2 main forces that drive change.
				Environment as a drive for adaptation.
				Society : safety network, cooperation and competition
\\
			\textbf{niche construction theory}
				AP occupy space and cause HG displacement.
\\
\\





	\section{What is SugarScape?}
	\section{Added advanced features}
		\paragraph{Same deduced trends and emerging dynamics}
		\paragraph{Realistic Adaptability to Parameter Perturbations}
	\section{Solving critics against classic SugarScape}
	\section{Experiments}

Made assumptions about behavior of real systems\\
  1st step, test if assumptions are reasonable\\
    -Validation, or representativeness of assumptions\\
  2nd step, test whether model implements assumptions\\
    -Verification, or correctness\\

La següent comprovació fes-la en el modelatge o un altre capitol més adient:\\
Seed independence – random number generator starting value should not affect final conclusion (maybe individual output, but not overall conclusion)
Three key aspects to validate:
  -Assumptions
  -Input parameter values and distributions
  -Output values and conclusions





		\subsection{Initial Conditions}
			\subsubsection{Montecarlo?} 
			\paragraph{Emergence of stationary state; initial state := stationary state}
		\subsection{Experiment features}
			\paragraph{Description}
			\paragraph{Hypothesis}
			\paragraph{Assumptions}
			\paragraph{Config}
			\paragraph{Results}
			\paragraph{Validation}

\newpage 
\chapter{Gujarat Case Modelization} % ODD casi a pelo, comprovar les policies MDP,programmed,etc...
	\section{Introduction}
Northern Gujarat is a marginal environment between the Thar Desert and the more fertile area of Saurashtra. This region is an ecotone, characterized by the seasonal influence of the monsoon where contrasting ecological niches are in tension and small climatic shifts can generate significant environmental changes, eventually affecting  resource availability. Archaeological evidence points to the presence and possible coexistence in the area of groups of people with different resource management strategies and mobility behaviors: hunter-gatherers (HG); agropastoralists (AP); urban Harappans (UH).
The aim of this study is to model resource management and decision making among hunter-gatherer groups in this region to explore adaptive trajectories and performance in relation to a) environmental variability and b) the appearance of other specialized groups . 
What factors play a role in HG persistence or disappearance in arid margins? Is the advent of agro-pastoral behaviour a big enough change to explain the disappearance of HG behaviour? What happens when there is an external influence, such as that by UH? Does climate change affect HG behaviour?
	    \subsection{Hypotheses}
In our starting hypothesis HG groups are adapted to marked seasonality (due to monsoon) in the arid margins of northern Gujarat. We intend to explore HG resilience considering: a) the appearance of AP, b) the appearance of an external attractor (UH) and c) climate change. We define resilience as the ability of the system to maintain its identity in the face of internal change and external perturbation (Carpenter 2001).
	    \subsection{Aims and objectives}
	    \subsection{Knowledge Elicitation \& Brainstorming}
		\subsubsection{Interviews}
		\subsubsection{ECOTONO (journal club)}
		\subsubsection{ODD}

	\section{Physical World / Environment}		% put this section in Gujarat Case Study???
		\subsection{Statistical Modelling}
			\subsubsection{Data Sources}
			\subsubsection{Resource Pipeline}

	\section{Antrophological Model}				% put this section in Gujarat Case Study???
		\subsection{The Model}
			\subsubsection{Knowledge Represent}
				Arithmetics, logics, probab models,... which \& why

			\subsubsection{Decission Process}
				\paragraph{Hypothesis:richer agents}
				\paragraph{UPF hand to hand work:UCT algorithm}
				\paragraph{Methods}
				\paragraph{¿state of the art?}

			\subsubsection{Social Network}
				\paragraph{¿state of the art?}

			\subsubsection{Design}
				\paragraph{Organisational level design}
				\paragraph{Social structure}
				\paragraph{Interaction structure}
				\paragraph{Communicative structure}
				\paragraph{Normative structure}

			\subsubsection{Coordination level design}
				\paragraph{Action model}
				\paragraph{Task model}
				\paragraph{Agent model}
				\paragraph{Plan model}
	\section{Experiments}
		\subsection{Initial Conditions}
			\subsubsection{Montecarlo?} 
			\paragraph{Emergence of stationary state; initial state := stationary state}
		\subsection{Experiment features}
			\paragraph{Description}
			\paragraph{Hypothesis}
			\paragraph{Assumptions}
			\paragraph{Config}
			\paragraph{Results}
			\paragraph{Validation}


\newpage 
\chapter{Conclusion}
      \section{Achieved Objectives}
      \section{Achieved Objectives}
      \section{Comparison AI - Simple}
      \section{Difficulties \& Issues}
      \section{Publications/CAA}
      \section{Future Issues}
\newpage 

\chapter{Bibliography}
\begin{thebibliography}{12}
\bibitem{Epstein1999} 
	J.M. Epstein. 
	\emph{Agent-based computational models and generative social science. Generative Social
	Science: Studies in Agent-Based Computational Modeling}, pages 4-46, 1999.
\bibitem{EpsteinAxtell}
	J.M. Epstein and R. Axtell.
	\emph{Growing Artificial Societies}, 1996.
\bibitem{Axelrod2007}
	R. Axelrod. 
	\emph{Simulation in Social Sciences. Handbook of research on nature-inspired computing for economics and management}, 1:90, 2007.
\bibitem{JDoran}
	Doran, J., 
	\emph{1999. Prospects for Agent-Based modelling in Archaeology}. Archeologia e Calcolatori, 10, 33-44.
\bibitem{GilbertABM}
	Gilbert, N.,
	\emph{2008. Agent-Based Models}. SAGE Publications, California.
\bibitem{GilbertTroitzsch}
	Gilbert, N., Troitzsch, K.G.,
	\emph{2008. Simulation for the Social Scientist}. Open University Press, USA.
\bibitem{Lake}
	Lake, M W,
	\emph{2000. Computer Simulation of Mesolithic Foraging, in: Gumerman, G. J., Kohler, T.A. (Eds.), Dynamics in Human and Primate Societies: Agent-Based Modeling of Social and Spatial Processes}, Oxford University Press, New York, pp. 107-143.
\bibitem{Sugarscape}
	Sugarscape.
	\emph{http://ccl.northwestern.edu/netlogo/models/community/Sugarscape}
\bibitem{Epstein2008}
	Joshua Epstein
	\emph{Why Model?}.
	2008.
\bibitem{Pidd2012}
	Pidd
	\emph{Pidd book}.
	2012.
\bibitem{Pidd2010}
	Pidd
	\emph{Pidd book}.
	2010.
\bibitem{Pidd2003}
	Pidd
	\emph{Pidd book}.
	2003.
\bibitem{Leeuw2004}
	Leeuw
	\emph{Leeuw book}.
	2004.
\bibitem{Robinson2008}
	Robinson
	\emph{Robinson book}.
	2008.
\bibitem{Balci1988}
	Balci
	\emph{Balci book}.
	1988.
\bibitem{Heath2009}
	Heath
	\emph{Heath book}.
	2009.
\bibitem{Banks1998}
	Banks
	\emph{Banks book}.
	1998.
\bibitem{Davies2003}
	Davies
	\emph{Davies book}.
	2003.
\bibitem{Axelrod1997a}
	Axelrod
	\emph{Axelrod book}.
	1997.
\end{thebibliography}

\end{document}
