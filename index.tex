
%\documentclass[man]{apa}
%\documentclass[doc]{article}{styles/apacls/apa}
\documentclass{report}


\usepackage{mathptmx}       % selects Times Roman as basic font
\usepackage{helvet}         % selects Helvetica as sans-serif font
\usepackage{courier}        % selects Courier as typewriter font
\usepackage{type1cm}        % activate if the above 3 fonts are
                            % not available on your system
\usepackage[utf8]{inputenc}
\usepackage{textcomp}
\usepackage{makeidx}         % allows index generation
\usepackage{graphicx}        % standard LaTeX graphics tool
                             % when including figure files
\usepackage{multicol}        % used for the two-column index
\usepackage[bottom]{footmisc}% places footnotes at page bottom

\usepackage{verbatim}        % per comentaris multilinia
\usepackage[utf8]{inputenc} %permite escribir {\'a},{\'e},{\'u},{\'o},{\'\i} & {\~n} directamente
\usepackage[british]{babel}
\usepackage{csquotes}
\usepackage{amsmath, amsthm, amssymb}


\begin{document}
\title{Coevolutionary strategies in MultiAgent systems. An approach using socionatural realistic environments.\\
\small{Master Thesis}}
\author{Alexis Torrano Mart\'inez}
\maketitle
\newpage

\pagenumbering{roman}
\begin{abstract}
The aim of this master thesis is the development of a multiagent model for a simulation  of two populations whose interactions are strongly influenced by a realistic landscape.
This research will be in line with Consolider-Simulpast (www.simulpast.es), an interdisciplinary project aimed to create simulations designed to be used in archaeological studies of human-environment interaction, decision-making processes and coevolutionary/competition behaviours of past societies.
The work plan will be focused on the development of first-stage models for two societies in the age of agriculture spreading surpasing the hunting and foraging way of living. The simulation will involve a climate engine for seasonality depending primarily on variable rainfall rate. Landscape information will be created from satellite image rasters. Constants, and variable relationship shall be modelled from measures and interviews with the experts. Data analysis tasks will be undertaken to validate the models and detect patterns in the archaeological record. Furthermore a comparison will be stablished between the classical simple models used in social simulation[1][2][8] and more advanced approaches.
\end{abstract}
\newpage 

\setcounter{tocdepth}{6}
\tableofcontents
\newpage 

\pagenumbering{arabic}
\chapter{Introduction}

	\section{Description}
		Problem, Gujarat, Archeology
		
	\section{Motivation}
			\paragraph{interaction society vs envirm}
			\paragraph{niche construction theory}
			
	\section{Simulation}
	
	\section{Question}
			\paragraph{Do AI techniques contribute to better simulation results?}
			\paragraph{Classic Simple Agent approach vs Rich Agents}
			\paragraph{Did Gujarat extreme enviromental conditions delayed the HG disappearence?}
			
\newpage 
\chapter{Methodology}
	\section{Intro}
%Dynamic systems tied to time. Non reducible to a formula, so the system must be replicated
%in a abstract calculus framework.  methodology with integration of system state, time component
%and repeatable experimentation for behaviour induction : simulation

%simulation : virtual experimentation in computers for a posteriori analisys of the dynamics.

%the system is reproduced in a model. The framework of the simulation reproduces the conditions
%for the experiments and iterates the changes of the system over time.



The question stated and the domain where that question is answered, demands a methodology that should cope with time related dynamics, with systems the change across time. When you ask something to a system, sometimes, you can solve it with a formula, a one step or time dimension independent number of steps calculus. Some other times you cannot jump and skip the behaviour of the system between an initial time and a final time where you think your answer is solved. The last ones are called irreducible systems. In such systems, to ask something about the state of the system in time $t_n$ you have to calculate all the states between an initial $t_0$ and that end state at $t_n$. So, you have to reproduce the dynamic of the system along the time. And that is what simulation is about. 
The Gujarat case, and in general social dynamics systems, fit well with the simulation approach. You have a system the changes along time. A system which is complex enough that does not allow to forget the middle states because for each new state you must induce it from the nearest past state.


The use of agent-based modelling and simulation techniques in the social sciences has flourished in the recent decades. 
The main reason is that the object of study in these disciplines, human society present or past, is difficult to analyze through classical analytical 
techniques. Population dynamics and structures are inhrently complex. Thus, other methodological techniques need to be found to more adequately study 
this field. In this context, agent-based modelling is encouraging the introduction of computer simulations to examine behavioural patterns in complex systems.
Simulation provides a tool to artificially examine socities, where a big number of actors with decision capacity coexist and interact.




	\section{Conceptual Framework}


		\subsection{Evolution}

Configurations $c_1$,...,$c_n$. Strategies $\sigma_1$,...,$\sigma_n$. Along time we study adaptation of different strategies. 

		\subsection{Coevolution}

Retro feedback of the different actors inside the evolution phenomena.
Everybody conditions the othe with its outcomes and success/fail in the evolution of
its adaptation drive.

			\subsubsection{Evolutionary \/ Genetic Programming?}
	\section{ABMs}
	  \subsection{Intro}
	  \subsection{ABM justification}		
	   \subsection{Platforms / Software packages}
		\subsubsection{NetLogo...}
		\subsubsection{Pandora/Cassandra}
	\section{Intelligent agents}
	  \subsection{Planners, ...}
\newpage 	
\chapter{SugarScape vs Advanced Sugarscape}
	\section{What is SugarScape?}
	\section{Added advanced features}
		\paragraph{Same deduced trends and emerging dynamics}
		\paragraph{Realistic Adaptability to Parameter Perturbations}
	\section{Solving critics against classic SugarScape}
	\section{Experiments}	
		\subsection{Initial Conditions}
			\subsubsection{Montecarlo?} 
			\paragraph{Emergence of stationary state; initial state := stationary state}
		\subsection{Experiment features}
			\paragraph{Description}
			\paragraph{Hypothesis}
			\paragraph{Assumptions}
			\paragraph{Config}
			\paragraph{Results}
			\paragraph{Validation}
	
\newpage 
\chapter{Gujarat Case Modelization}
	\section{Introduction}
Northern Gujarat is a marginal environment between the Thar Desert and the more fertile area of Saurashtra. This region is an ecotone, characterized by the seasonal influence of the monsoon where contrasting ecological niches are in tension and small climatic shifts can generate significant environmental changes, eventually affecting  resource availability. Archaeological evidence points to the presence and possible coexistence in the area of groups of people with different resource management strategies and mobility behaviors: hunter-gatherers (HG); agropastoralists (AP); urban Harappans (UH).
The aim of this study is to model resource management and decision making among hunter-gatherer groups in this region to explore adaptive trajectories and performance in relation to a) environmental variability and b) the appearance of other specialized groups . 
What factors play a role in HG persistence or disappearance in arid margins? Is the advent of agro-pastoral behaviour a big enough change to explain the disappearance of HG behaviour? What happens when there is an external influence, such as that by UH? Does climate change affect HG behaviour?
	    \subsection{Hypotheses}
In our starting hypothesis HG groups are adapted to marked seasonality (due to monsoon) in the arid margins of northern Gujarat. We intend to explore HG resilience considering: a) the appearance of AP, b) the appearance of an external attractor (UH) and c) climate change. We define resilience as the ability of the system to maintain its identity in the face of internal change and external perturbation (Carpenter 2001).
	    \subsection{Aims and objectives}
	    \subsection{Knowledge Elicitation \& Brainstorming}
		\subsubsection{Interviews}
		\subsubsection{ECOTONO (journal club)}	
		\subsubsection{ODD}

	\section{Physical World / Environment}		% put this section in Gujarat Case Study???
		\subsection{Statistical Modelling}
			\subsubsection{Data Sources}
			\subsubsection{Resource Pipeline}
			
	\section{Antrophological Model}				% put this section in Gujarat Case Study???
		\subsection{The Model}
			\subsubsection{Knowledge Represent}
				Arithmetics, logics, probab models,... which \& why
				
			\subsubsection{Decission Process}
				\paragraph{Hypothesis:richer agents}
				\paragraph{UPF hand to hand work:UCT algorithm}
				\paragraph{Methods}
				\paragraph{¿state of the art?}
				
			\subsubsection{Social Network}
				\paragraph{¿state of the art?}

			\subsubsection{Design}
				\paragraph{Organisational level design}
				\paragraph{Social structure}
				\paragraph{Interaction structure}
				\paragraph{Communicative structure}
				\paragraph{Normative structure}
				
			\subsubsection{Coordination level design}
				\paragraph{Action model}
				\paragraph{Task model}
				\paragraph{Agent model}
				\paragraph{Plan model}	
	\section{Experiments}	
		\subsection{Initial Conditions}
			\subsubsection{Montecarlo?} 
			\paragraph{Emergence of stationary state; initial state := stationary state}
		\subsection{Experiment features}
			\paragraph{Description}
			\paragraph{Hypothesis}
			\paragraph{Assumptions}
			\paragraph{Config}
			\paragraph{Results}
			\paragraph{Validation}
	

\newpage 
\chapter{Conclusion}
      \section{Achieved Objectives}
      \section{Achieved Objectives}
      \section{Comparison AI - Simple}
      \section{Difficulties \& Issues}
      \section{Publications/CAA}
      \section{Future Issues}
\newpage 
\chapter{Bibliography}
\begin{thebibliography}{9}
\bibitem{[1]} 
J.M. Epstein. 
\emph{Agent-based computational models and generative social science. Generative Social
Science: Studies in Agent-Based Computational Modeling}, pages 4-46, 1999.
\bibitem{[2]} 
J.M. Epstein and R. Axtell.
\emph{Growing Artificial Societies, 1996.}
\bibitem{[3]} 
R. Axelrod. 
\emph{Simulation in Social Sciences. Handbook of research on nature-inspired computing for economics and management}, 1:90, 2007.
\bibitem{[4]}
Doran, J., 
\emph{1999. Prospects for Agent-Based modelling in Archaeology}. Archeologia e Calcolatori, 10, 33-44.
\bibitem{[5]}
Gilbert, N.,
\emph{2008. Agent-Based Models}. SAGE Publications, California.
\bibitem{[6]}
Gilbert, N., Troitzsch, K.G.,
\emph{2008. Simulation for the Social Scientist}. Open University Press, USA.
\bibitem{[7]}
Lake, M W,
\emph{2000. Computer Simulation of Mesolithic Foraging, in: Gumerman, G. J., Kohler, T.A. (Eds.), Dynamics in Human and Primate Societies: Agent-Based Modeling of Social and Spatial Processes}, Oxford University Press, New York, pp. 107-143.
\bibitem{[8]}
Sugarscape.
\emph{http://ccl.northwestern.edu/netlogo/models/community/Sugarscape}
\end{thebibliography}

\end{document}
